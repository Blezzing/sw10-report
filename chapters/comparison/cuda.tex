\section{CUDA}
The evaluation of \textit{CUDA} is contained in this section. First the code for the demo application is shown and explained, followed by the general evaluation and the usability evaluation. The last subsection is a summary of the evaluation of \textit{CUDA} which will serve as a baseline in the comparison in section \ref{sec:comparisoncomparison}.

\subsection{Code of demo application}

\begin{lstlisting}[caption={\textit{CUDA} \textit{SAXPY} example.}, label={code:cudaSaxpy}]
#include <iostream>

__global__ void kernel(int n, float a, float* x, float* y){
    for( int i = blockIdx.x * blockDim.x + threadIdx.x; i < n; i += blockDim.x * gridDim.x){
        x[i] = a * x[i] + y[i];
    }
}

int main(void){
    int N = 1 << 29;
    float a = 11.0;

    float *x, *y, *d_x, *d_y;
    x = (float*)malloc(N*sizeof(float));
    y = (float*)malloc(N*sizeof(float));

    cudaMalloc(&d_x, N*sizeof(float));
    cudaMalloc(&d_y, N*sizeof(float));

    for(int i = 0; i < N; i++){
        x[i] = 1;
        y[i] = 2;
    }

    cudaMemcpy(d_x, x, N*sizeof(float), cudaMemcpyHostToDevice);
    cudaMemcpy(d_y, y, N*sizeof(float), cudaMemcpyHostToDevice);

    kernel<<<128, 128>>>(N, a, d_x, d_y);

    cudaMemcpy(x, d_x, N*sizeof(float), cudaMemcpyDeviceToHost);
    cudaMemcpy(y, d_y, N*sizeof(float), cudaMemcpyDeviceToHost);
}

\end{lstlisting}

\subsection{General Evaluation}

Performance: 

Lines of code: 

Size of executable: 

\subsection{Usability Evaluation}

\subsubsection[*]{Viscosity}

\subsubsection[*]{Visibility}

\subsubsection[*]{Premature Commitment}

\subsubsection[*]{Hidden Dependencies}

\subsubsection[*]{Role-expressiveness and Consistency}

\subsubsection[*]{Error-proneness}

\subsubsection[*]{Abstraction}

\subsubsection[*]{Closeness of mapping}

\subsubsection[*]{Diffuseness}

\subsubsection[*]{Hard Mental Operations}

\subsubsection[*]{Progressive Evaluations}

\subsection{Summary}
