\section{Thrust}
The evaluation of \textit{Thrust} is contained in this section. First the code for the demo application is shown and explained, followed by the general evaluation and the usability evaluation. The last subsection is a summary of the evaluation of \textit{Thrust} which will be used in the comparison in section \ref{sec:comparisoncomparison}.

\subsection{Code of demo application}

\begin{lstlisting}[caption={\textit{Thrust} \textit{SAXPY} example.}, label={code:thrustSaxpy}]
size_t N = 1024;
float a = 10;

//cuda classifier on lambda
auto func = [=]__device__(float x, float y){return a * x + y;};

//initialize host vectors
thrust::host_vector<float> x(N);
thrust::host_vector<float> y(N);
thrust::host_vector<float> z(N);

//fill with random data
std::generate(x.begin(), x.end(), rand);
std::generate(y.begin(), y.end(), rand);

//copy to device
thrust::device_vector<float> d_x = x;
thrust::device_vector<float> d_y = y;

//perform saxpy
thrust::transform(d_x.begin(), d_x.end(), d_y.begin(), d_y.begin(), func); ~\label{code:thrustSaxpy:execute}~

//copy results back to host vector
z = d_y;
\end{lstlisting}

\subsection{General Evaluation}

Performance: 

Lines of code: 

Size of executable: 

\subsection{Usability Evaluation}

\subsubsection[*]{Viscosity}

\subsubsection[*]{Visibility}

\subsubsection[*]{Premature Commitment}

\subsubsection[*]{Hidden dependencies}

\subsubsection[*]{Role-expressiveness}

\subsubsection[*]{Error-proneness}

\subsubsection[*]{Abstraction}

\subsubsection[*]{Closeness of mapping}

\subsubsection[*]{Consistency}

\subsubsection[*]{Diffuseness}

\subsubsection[*]{Hard mental operations}

\subsubsection[*]{Progressive evaluations}

\subsection{Summary}
