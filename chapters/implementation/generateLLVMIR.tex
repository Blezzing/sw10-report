\section{Generation of LLVM IR}
 We generate \textit{LLVM Intermediate Representation}, which we now refer to as \textit{IR}, which can later be transformed to \textit{PTX} code, and be executed. We use the tools provided by the \textit{LLVM} library to generate this \textit{IR}, as they manage details such as variable names and references.

The \textit{IR} is based on a nested structure as seen on \ref{fig:llvmTypes}, where \texttt{LLVM Module} is the program, \texttt{LLVM Function} is a function that is a member of the module,  \texttt{LLVM Basic Block} is a code block, and \texttt{LLVM Instruction} is a instruction. 

\begin{figure}[!htb]
    \center
    \label{fig:llvmTypes}
    \includegraphics[width=0.65\linewidth]{chapters/implementation/figs/llvmTypes.png}
    \caption{The LLVM IR hierachy.}
\end{figure}

When a \texttt{yagal::Vector}'s \texttt{exec} function is called, the queued actions is used as source for the \textit{IR} generation. The procedure for \textit{IR} generation is as follows:

\begin{enumerate}
\item Generate \textit{LLVM Module} with platform information.
\item Generate \textit{LLVM Function} with parameter information extracted from the set of action on the \texttt{Vector}.
\item Generate \textit{LLVM Basic Block} to perform logic of a single action.
\item Repeat step 3 until all actions have a corresponding \textit{LLVM Basic Block}.
\item Connect \textit{LLVM Basic Block}s by branching instructions to create program flow.
\item Generate \textit{LLVM Metadata} for functions.
\end{enumerate}

This procedure have the benefit of allowing multiple actions to be compiled to a single kernel function, and as such can be executed on the GPU in a single step.

The implementation of the procedure is presented in the following subsections.

\subsection{Generating a LLVM Module} \label{sec:irmodule}
The \textit{LLVM Module} is the outer most abstraction of \textit{IR}. In our implementation we encapsulate it in the class \texttt{IRModule}. This class contains functions needed for us to generate \textit{IR}, and have the \textit{LLVM Module} as a member variable that is accessed through these functions. A listing showing an extract of the class with the relevant content can be seen in listing \ref{code:irModule}.

Before we can create any \textit{LLVM} objects, a \texttt{LLVMContext} is needed. A \texttt{LLVMContext} is a container of the state that \textit{LLVM} is in, and is declared at line \ref{code:irModule:context}.

With the \texttt{context} we create the module in the constructor chain at line \ref{code:irModule:module}. An architecture must then be defined for the module. We provide the string \texttt{"nvptx64-nvidia-cuda"}, as it is a key in the \textit{LLVM} library, that is used to look up target information during code generation. 

\begin{lstlisting}[caption={The IRModule class}, label={code:irModule}]
namespace yagal::generator{
    //Representation of a module in llvm ir.
    //Contains the logic to configure a llvm module and provide functions for the rest of the library to add functionality to a module
    class IRModule{
    public:
        /* Some fields omitted */

        llvm::LLVMContext context; ~\label{code:irModule:context}~
        llvm::Module module;

        uint64_t elementsToHandle;

        llvm::Function* getThreadIdxIntrinsic;
        llvm::Function* getGridDimxIntrinsic;


        IRModule(uint64_t numberOfElements): 
            module("yagalModule", context), ~\label{code:irModule:module}~
            getThreadIdxIntrinsic(llvm::Intrinsic::getDeclaration(&module, llvm::Intrinsic::nvvm_read_ptx_sreg_tid_x)),
            getGridDimxIntrinsic(llvm::Intrinsic::getDeclaration(&module, llvm::Intrinsic::nvvm_read_ptx_sreg_nctaid_x)),
            elementsToHandle(numberOfElements)
        {
            //Set platform specific variables for the module.
            module.setTargetTriple("nvptx64-nvidia-cuda");

            _p.debug() << "ir module constructed" << std::endl;
        }

        //Create a function ready for insertion with a IRBuilder.
        llvm::Function* createKernel(int numberOfParameters){
            /* omitted */
        }

        //Creates the return point of a kernel, and links blocks together, to effectively make them labels.
        void finalizeKernel(llvm::Function* kernel){
            /* omitted */
        }

        //Update metadata of module to correctly tag the kernel functions.
        void updateMetadata(){
            /* omitted */
        }

        /* other functions omitted */
    };
}
\end{lstlisting}

With the module created, we can start building \textit{LLVM Function}s to create our kernel.

\subsection{Generating a LLVM Function} \label{sec:irFunction}
A \textit{LLVM Function} can be mapped to a GPU kernel, as a GPU kernel is a function callable on the GPU. The code we use to generate an \textit{IR} function is shown in listing \ref{code:createKernelInitial}. 

Before we can declare the function, we need to prepare argument types, as seen from line \ref{code:createKernelInitial:arg:begin} to \ref{code:createKernelInitial:arg:begin} by providing type information and address space of parameters. The second parameter to \texttt{getFloatPtrTy}, \texttt{1} indicates which address space the parameter is in, with 1 being the global memory of the device.

We construct the function as a member of the module on line \ref{code:createKernelInitial:func}, by providing type information, linking method for external functions, name, and containing module. The type information consist of the return type \texttt{void}, as we expect no direct return value from a kernel, the previously constructed parameter types, and whether the number of parameters can vary. The linkage refers to the visibility of the function, with external meaning that it can be accessed from other modules.

To make sure the function follows the calling convention of the platform, we use the \texttt{setCallingConvention} function, as seen on line \ref{code:createKernelInitial:cc}. This means that parameter parsing is done according to convention for us.

\begin{lstlisting}[caption={The createKernel function}, label={code:createKernelInitial}]
llvm::Function* createKernel(int numberOfParameters){
    std::vector<llvm::Type *> kernel_arg_types; ~\label{code:createKernelInitial:arg:begin}~
    for(int i = 0; i < numberOfParameters; i++){
        kernel_arg_types.push_back(llvm::Type::getFloatPtrTy(context, 1));
    } ~\label{code:createKernelInitial:arg:end}~

    auto kernel = llvm::Function::Create( ~\label{code:createKernelInitial:func}~
        llvm::FunctionType::get(llvm::Type::getVoidTy(context), kernel_arg_types, false),
        llvm::Function::ExternalLinkage,
        llvm::Twine("kernel"),
        &module
    );

    kernel->setCallingConv(llvm::CallingConv::PTX_Kernel); ~\label{code:createKernelInitial:cc}~

    return kernel;
}
\end{lstlisting}

With the function created we can start building \textit{LLVM Basic Block}s in them, to provide functionality to the kernel.

\subsection{Generating a LLVM Basic Block Based On an Action} \label{sec:bbbuilding}
The functionality needed of a kernel is dictated by the actions described in section \ref{cha:queueingActions}. All actions have a function that generates a corresponding \textit{LLVM Basic Block}. This function can be seen on line \ref{code:addAction:func} in listing \ref{code:addAction}.

First we create a \textit{LLVM Basic Block} at line \ref{code:addAction:bbc} in listing \ref{code:addAction}, by providing the \texttt{LLVM Context} of the \textit{IR} container, a name, and the function that should contain it. We then create a \texttt{LLVM IRBuilder}, which is a object that provides functions for generating \textit{IR} from \textit{LLVM} objects, and adds the block to the vector of blocks, that should be managed by the \textit{IR} container.

After constructing the \textit{LLVM Basic Block}, we get a pointer to the first kernel parameter, which is the vector we are manipulating. We can now start building the \textit{IR}.

We use the \texttt{IRBuilder} to create the following instructions as seen in listing \ref{code:addAction}:
\begin{itemize}
\item Line \ref{code:addAction:aload} creates the \texttt{load} instruction, that load the index of the value that should be modified.
\item Line \ref{code:addAction:gep} creates the \texttt{getElementPtr} instruction, that get the pointer to the correct element of the vector based on index.
\item Line \ref{code:addAction:tmp} creates the \texttt{load} instruction, that load the value of the vector before modification.
\item Line \ref{code:addAction:const} creates a constant value based on the value the AddAction was constructed with.
\item Line \ref{code:addAction:add} creates the \texttt{add} instruction for floating point values, that adds the constant to the loaded value, and assigns a temporary value for the result.
\item Line \ref{code:addAction:astore} creates the \texttt{store} instruction, that stores the temporary value at the address of the element pointer.
\end{itemize}

\begin{lstlisting}[caption={The AddAction class}, label={code:addAction}]
template <typename T>
class AddAction : public SimpleAction<T>{
public:
    AddAction(T v): SimpleAction<T>(v) {}

    void generateIR(yagal::generator::IRModule& ir, llvm::Function* kernel, int& inputVectorCounter){ ~\label{code:addAction:func}~
        _p.debug() << "generateIR for add action called." << std::endl;

        //Prepare the block to fill in
        auto actionBlock = llvm::BasicBlock::Create(ir.context, llvm::Twine(ir.getNextBasicBlockName()), kernel); ~\label{code:addAction:bbc}~
        llvm::IRBuilder<> builder(actionBlock);
        ir.userBlocks.push_back(actionBlock);

        //Prepare argument
        auto vecVal = kernel->arg_begin();
        vecVal->setName("vec");

        //Build llvm ir
        int alignment = 4;
        auto indexVar = builder.CreateAlignedLoad(ir.currentIndexValue, alignment, "i");                    ~\label{code:addAction:aload}~
        auto ptrVal = builder.CreateGEP(vecVal, indexVar, "ptr");                                           ~\label{code:addAction:gep}~
        auto tmpVal = builder.CreateAlignedLoad(ptrVal, alignment, "tmp");                                  ~\label{code:addAction:tmp}~
        auto inputConst = llvm::ConstantFP::get(llvm::Type::getFloatTy(ir.context), (float)this->value);    ~\label{code:addAction:const}~
        auto retVal = builder.CreateFAdd(tmpVal, inputConst, "ret");                                        ~\label{code:addAction:add}~
        builder.CreateAlignedStore(retVal, ptrVal, alignment);                                              ~\label{code:addAction:astore}~
    }
};
\end{lstlisting}

Now that we can build a basic block we are almost ready to execute the code. There still is no return statement in the function, and as such the code is still invalid as not all possible execution paths return.

\subsection{Conecting LLVM Basic Blocks} \label{sec:bblinking}
Now that we can generate \textit{LLVM Basic Block}s in a function, we need to go though all basic blocks of a function and ensure that they are connected in a way that lead to the function executing all basic blocks and returning when done.

To control the flow, two additional basic blocks are created, one representing the entry, and one representing the exit. The entry block contains an unconditional branch to the first basic block, and the exit block contains the return statement. We then iterate all basic blocks in the function, and add a branch instruction at the end, leading to the next basic block. When we reach the last basic block, we make it branch to the exit block. Figure \ref{fig:bbflow} shows the resulting flow, and with that we are done modifying the content of the function.

\begin{figure}[!htb]
    \center
    \label{fig:bbflow}
    \includegraphics[width=0.65\linewidth]{chapters/implementation/figs/bbflow.png}
    \caption{The connections of Basic Blocks}
\end{figure}

\subsection{Generating LLVM Metadata} \label{sec:metadata}
The \textit{LLVM Function} is ready to be executed on the GPU, but is not marked as a GPU kernel yet. The \textit{NVPTX} back end requires us to provide the function we want marked as a kernel, and to do this we use \textit{LLVM Metadata}. On listing \ref{code:updateMetadata} we show how we use the \textit{LLVM API} to create a metadata node tagging our function, named "kernel", as a kernel function.

\begin{lstlisting}[caption={The updateMetadata function}, label={code:updateMetadata}]
void updateMetadata(){
    auto metadata = module.getOrInsertNamedMetadata("nvvm.annotations");
    auto oneconstant = llvm::ConstantInt::get(llvm::Type::getInt32Ty(context), 1);
    std::vector<llvm::Metadata *> ops{
        llvm::ValueAsMetadata::getConstant(module.getNamedValue("kernel")),
        llvm::MDString::get(context, "kernel"),
        llvm::ValueAsMetadata::getConstant(oneconstant)
    };
    auto metadata_node = llvm::MDTuple::get(context, ops); 
    metadata->addOperand(metadata_node);
}
\end{lstlisting}

\subsection{Putting The Steps Together}\label{cha:puttingStepsTogether}
Now that we have all necessary components for building \textit{IR}, we can put it together. We do this in the \texttt{exec} function on the \texttt{yagal::Vector<T>}.

At line \ref{code:exec:construct} in Listing \ref{code:exec} we construct the \texttt{IRModule}, which will construct a \textit{LLVM Module} ready for \textit{LLVM Function} insertion, as described in section \ref{sec:irmodule}.

At line \ref{code:exec:param:begin} to \ref{code:exec:param:end} we prepare the parameters to kernel function. We do this by collecting the \texttt{CUdeviceptr}s of all vectors required by actions queued on the vector, so they are ready to be provided to the kernel.

At line \ref{code:exec:bb:begin} to \ref{code:exec:bb:end} we generate a \textit{LLVM Basic Block} as described in section \ref{sec:irFunction}, and creates a basic block for each action in this function as described in section \ref{sec:bbbuilding}.

At line \ref{code:exec:link} we link the basic blocks of the function together, to complete the function, as described in section \ref{sec:bblinking}.

At line \ref{code:exec:meta} we create the metadata needed to let the compiler identify the function as a kernel, as described in section \ref{sec:metadata}

At line \ref{code:exec:translate} we can translate the constructed \textit{IR} to \textit{PTX}, and be ready to execute it. How the translation is performed is covered in section \ref{sec:llcReplacement}.

At line \ref{code:exec:ex} we call the function described in section \ref{cha:execPtx}, to execute the kernel with our parameters.

At line \ref{code:exec:clean} we empty the action vector to be ready for a new action queue on the same vector.

\begin{lstlisting}[caption={The exec function}, label={code:exec}]
Vector<T>& exec(){
    yagal::generator::IRModule ir(_count); ~\label{code:exec:construct}~

    //Count number of cuda parameters needed, starting at 1 to include the vector itself.
    std::vector<CUdeviceptr*> devicePointers({&_devicePtr}); ~\label{code:exec:param:begin}~
    for (auto& a : _actions){
        if(a->requiresCudaParameter()){
            auto pa = static_cast<internal::ParameterAction<T>*>(a.get());
            auto ptr = pa->getDevicePtrPtr();
            devicePointers.push_back(ptr);
        }
    }  ~\label{code:exec:param:end}~

    //Generate llvm ir blocks.
    int inputVectorCounter = 0; ~\label{code:exec:bb:begin}~
    auto kernel = ir.createKernel(devicePointers.size());
    for (const auto& a : _actions){
        a->generateIR(ir, kernel, inputVectorCounter);
    }  ~\label{code:exec:bb:end}~

    //Link blocks and update metadata.
    ir.finalizeKernel(kernel);~\label{code:exec:link}~
    ir.updateMetadata();~\label{code:exec:meta}~

    //Generate code
    _p.debug() << ir.toString() << std::endl;
    yagal::generator::PTXModule ptx(ir); ~\label{code:exec:translate}~
    auto ptxSource = ptx.toString();
    _p.debug() << ptx.toString() << std::endl;
    
    //Execute kernel
    yagal::cuda::executePtxWithParams(ptxSource, devicePointers);  ~\label{code:exec:ex}~

    //Cleanup
    _actions.clear(); ~\label{code:exec:clean}~

    return *this;
}
\end{lstlisting}

This function is one of the fundamental elements of \textit{YAGAL} as it handles the flow of action execution.

%baah We construct a module for each computation, with a single function representing the kernel function, that need to be executed on the gpu. Inside this function we construct basic blocks that each perform a single action, based on the