\section{Matching Number of Threads with Vector Elements}
As described in section \ref{sec:actionToIr}, we can generate \textit{PTX} code from actions to modify vectors. There is however a problem which is not handled at this point: The element modified by a kernel is chosen, by using the kernels index as index in the vector. This mean that in cases of a vector having more elements than threads, the execution would not modify all values in the vector, and in the case of more threads than elements, the execution would go out of bounds, possibly corrupting other data.

The solution to this problem is to implement logic for the kernels that prevents execution of logic on elements that are out of bounds, and ensure that all elements in a vector will be modified exactly once.

With threads being organized in blocks, and blocks being organized in a grid, we use a method called striding, where each thread of each block handle a specific index in the vector before jumping to the next dedicated index. This logic is implemented to follow the process:

\begin{enumerate}
\item Set current index value to $threadIdX + blockDimX * blockIdX$.
\item If the index value is valid in the vector, handle logic, otherwise exit.
\item Increment the current index value by $blockDimX * gridDimX$.
\item repeat from step 2.
\end{enumerate}

We only consider the \textit{x} dimension, although we can use up to three, as a \texttt{yagal::Vector} only have a size in a single dimension.

We implement this feature, by extending any function we generate with two \textit{LLVM Basic Block}s and extending the entry block. The entry block will have the logic for the initialization of the index value, and branch to the first new block, rather than the first user block. The first new block is for checking whether the condition of the index being valid is met, and branch based on that to either the user block chain or the exit block. The second new block is for incrementing the index value by the number of threads, and branch to the condition block. The last user block will be changed to branch to the increment block, rather than the exit block. This implementation is illustrated on figure \ref{blockFlowLoop}, where it is shown how the different blocks are connected.

\begin{figure}[!htb]
    \centering
    \includegraphics[width=1\linewidth]{chapters/implementation/figs/IRDiagram.png}
    \caption{How blocks are connected within our generated \textit{LLVM IR}.}
    \label{blockFlowLoop}
\end{figure}