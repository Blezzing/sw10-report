\section{Queuing Actions} \label{cha:queueingActions}
To modify a \texttt{yagal::Vector}, we create a concept of actions. These actions contain the necessary information for generating a kernel, that executes the corresponding logic.

An action is an intended task to be performed on the \texttt{yagal::Vector}, and is constructed as a result of a developer calling a function on the \texttt{yagal::Vector}. For instance, calling \texttt{myVector.add(5)} will result in an \texttt{AddAction} being created with a copy of the parameters, \texttt{5} in this example, needed to perform the calculation. The developer must then call \texttt{myVector.exec()} to perform all applied actions. We use the actions as the source of our code generation, as we generate code to perform exactly the content of the action.

The template class \texttt{Vector<T>} is expanded, as seen in Listing \ref{code:actionAdditions}. A vector of \texttt{Action<T>} elements are added to the fields at line \ref{code:actionAdditions:vec}, it contains the actions needing to be performed on the \texttt{Vector<T>}, in the order they were added. Functions have been added to generate actions that are placed in this vector. An example of this is \texttt{Vector<T>\& add(T value)} at line \ref{code:actionAdditions:val}, which creates an action that represents adding a value to all elements in the \texttt{Vector<T>}, and places it at the back of the \texttt{\_actions} vector. The other added function is \texttt{Vector<T>\& exec()}, which is the function that consumes the action vector, generates \textit{LLVM IR} based on the consumed actions, translates the \textit{LLVM IR} to \textit{PTX}, loads the \textit{PTX} on the device, and executes it. The content of the \texttt{Vector<T>\& exec()} function is not the focus of this section, and is explained in section \ref{cha:puttingStepsTogether}.

\begin{lstlisting}[caption={Vector<T> action additions.}, label={code:actionAdditions}]
namespace yagal{
    template <typename T>
    class Vector{
    private:
        /* omitted fields */

        std::vector<std::shared_ptr<internal::Action>> _actions; ~\label{code:actionAdditions:vec}~

        /* omitted functions */
    
    public:
        Vector<T>& add(T value) {~\label{code:actionAdditions:val}~
            _actions.emplace_back(new internal::AddAction<T>(value));
            return *this;
        }
        
        /* omitted functions */

        Vector<T>& exec(){
            /* omitted logic */
        }
    }
}
\end{lstlisting}

Figure \ref{fig:actionClassDiagram} contain a class diagram of the inheritance for actions. The template class \texttt{Action<T>} is the single top level class in the hierarchy and is shown at the top of the class diagram. The every bottom level class is a concrete action that can be performed on a \texttt{yagal::Vector<T>} and are shown at the bottom of the class diagram as \texttt{AddAction} and \texttt{AddVectorAction}. These actions are grouped by mid level classes, that are shown in the class diagram as \texttt{SimpleAction} and \texttt{ParameterAction}, which define the input parameters. An example is the \texttt{AddAction<T>} being a \texttt{SimpleAction<T>}, as it takes a single input value to perform the action on an element of the \texttt{yagal::Vector<T>}, and the \texttt{SimpleAction<T>} being an \texttt{Action<T>} to allow us to contain it with other actions in the vector on line \ref{code:actionAdditions:vec}.

\begin{figure}[!htb]
    \centering
    \includegraphics[width=0.7\linewidth]{chapters/implementation/figs/actionClassDiagram.png}
    \caption{Class diagram for \textit{Action}.}
    \label{fig:actionClassDiagram}
\end{figure}
