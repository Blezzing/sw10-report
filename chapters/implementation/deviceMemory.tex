\section{Management of Device Memory}
We have created the template class \texttt{yagal::Vector<T>} to represent a vector of elements of type T, where the data is allocated on the device. This class is constructed to follow the idea of Section \ref{memoryModelDesign}, where we state that we want the developer to have explicit control of where memory is located. A code snippet with a stripped down version of the class, only showing the relevant content, is shown in Listing \ref{code:memoryManagementVector}, and line number references in this section refer to that listing.

The \texttt{yagal::Vector} class contains a device pointer that is used by the \textit{CUDA} driver API, and an integer that indicates the number of elements stored beginning at that address. The lifetime of the object is directly related to the lifetime of the data. When the object is created the necessary allocation is made on the device through the \textit{CUDA} driver API, and when the object is destroyed the allocated data is freed.

\texttt{yagal::Vector} uses an \textit{STL} \texttt{std::vector<T>} to provide a constructor and casting options for data transfers. The constructor copies data from an \textit{STL} \texttt{std::vector} to our \texttt{yagal::Vector} with \textit{CUDA}s \texttt{memcpy} function as a part of the object initialization, as shown at Line \ref{code:memoryManagementVectorConstructors}. To transfer data back we overload the casting operator to an \textit{STL} \texttt{std::vector}, copying the data from the device to the result \textit{STL} \texttt{std::vector} as seen at Line \ref{code:memoryManagementVectorCast}.

Accessing single data elements is done through \texttt{get} and \texttt{set} functions, as seen at Line \ref{code:memoryManagementVectorAcc}. These functions perform the needed memory transfer for the developer. They are intended only for tasks regarding a single value, as a transfer is needed for every use, and manipulations of the entire vector should be done either in main memory by the CPU, or in device memory by the GPU.

We chose not to implement an iterator for the \texttt{yagal::Vector} in order to avoid motivating CPU computations on device memory. This makes it different from an \textit{STL} \texttt{std::vector}, but iterating over elements can still be done, by copying the data from the device to an \textit{STL} \texttt{std::vector}.

\todo{figur der viser data placering måske..}

\todo{resize discussion, let's go?!?}

\begin{lstlisting}[caption={Vector class, showing only code relevant to memory management.}, label={code:memoryManagementVector}, mathescape]
namespace yagal{
    template<typename T>
    class Vector{
    private:
        CUdeviceptr _devicePtr;
        size_t _count;

    public
        // Constructors ~\label{code:memoryManagementVectorConstructors}~
        Vector(int elementCount)
            : _count(elementCount)
        {
            _devicePtr = yagal::cuda::malloc(_count * sizeof(T));
        }

        Vector(const std::vector<T>& source)
            : _count(source.size())
        {
            _devicePtr = yagal::cuda::malloc(_count * sizeof(T));
            yagal::cuda::copyToDevice(_devicePtr, source.data(), _count * sizeof(T));
        }

        // Cast (Copy out) ~\label{code:memoryManagementVectorCast}~
        operator std::vector<T>(){
            std::vector<T> result(_count);
            yagal::cuda::copyToHost(result.data(), _devicePtr, _count * sizeof(T));
            return result;
        }

        // Accessors ~\label{code:memoryManagementVectorAcc}~
        T getElement(int index){
            T result;
            yagal::cuda::copyToHost(&result, _devicePtr + (index * sizeof(T)), sizeof(T));
            return result;
        }

        void setElement(int index, T value){
            yagal::cuda::copyToDevice(_devicePtr + (index * sizeof(T)), &value, sizeof(T));
        }

        // Destructors
        $~$Vector(){
            yagal::cuda::free(_devicePtr);
        }

    }
}
\end{lstlisting}
\todo{fix tilde for destructor i eksempel}