\section{Execute PTX}

% Motivation
In order for \textit{YAGAL} to make use of the GPU, it must have a mechanism to execute code on it. Since \textit{LLVM} generates \textit{PTX} code, \textit{YAGAL} must be able to run it. This can be done by passing the \textit{PTX} to the \textit{CUDA} API. The \textit{CUDA} related handling have been contained within the \texttt{yagal::cuda} namespace. 

% Implementation
To execute the \textit{PTX} code, it is passed to the function \texttt{executePtxWithParams} as a string, along with a vector containing the input parameters as seen on Listing \ref{code:executePtxWithParams}. The line number references in this section refers this listing.

To ensure that the GPU is ready, the \texttt{initIfNeeded()} function at line \ref{code:executePtxWithParams:init} is called which creates a \textit{CUDA} context if one does not already exists.

At line \ref{code:executePtxWithParams:crtMd} a \textit{CUDA} module is created by calling \texttt{cuModuleLoadDataEx()}, which is part of the \textit{CUDA C} API. It takes \textit{PTX} code as a \texttt{C-string}, and outputs to the provided module container.

A function handle is needed in order to determine which function within the module to execute. The handle is identified by calling \texttt{cuModuleGetFunction()} at line \ref{code:executePtxWithParams:handle}. Here a function called \texttt{kernel} is requested from within the module. \texttt{kernel} is the hardcoded name of a function that is created by \textit{YAGAL}.

Predetermined block and grid dimensions are set through line \ref{code:executePtxWithParams:cfgStart} to \ref{code:executePtxWithParams:cfgEnd}. This part will be refactored such that dimensions are based upon the data instead of being hardcoded.
\todo{Remember to refactor this}

The module is executed on the GPU at line \ref{code:executePtxWithParams:launch} by a call to the function \texttt{cuLaunchKernel} and the module is then unloaded at line \ref{code:executePtxWithParams:clean} by \texttt{cuModuleUnload()}. The module have now been executed and the functionality within it is not needed anymore, therefore the module is unloaded.

\begin{lstlisting}[caption={int executePtxWithParams()}, label={code:executePtxWithParams}]
CUcontext  cudaContext;

int executePtxWithParams(const std::string& ptx, std::vector<CUdeviceptr*>& kernelParams){
    CUmodule    cudaModule;
    CUfunction  function;
    CUlinkState linker;
    int         devCount;

    initIfNeeded(); ~\label{code:executePtxWithParams:init}~

    // Create module for object
    checkCudaErrors(
       cuModuleLoadDataEx(&cudaModule, ptx.c_str(),0, 0, 0)); ~\label{code:executePtxWithParams:crtMd}~

    // Get kernel function
    checkCudaErrors(
        cuModuleGetFunction(&function, cudaModule, "kernel")); ~\label{code:executePtxWithParams:handle}~

    // Set kernel configuration
    unsigned blockSizeX = 16;  ~\label{code:executePtxWithParams:cfgStart}~
    unsigned blockSizeY = 1;
    unsigned blockSizeZ = 1;
    unsigned gridSizeX  = 1;
    unsigned gridSizeY  = 1;
    unsigned gridSizeZ  = 1; ~\label{code:executePtxWithParams:cfgEnd}~

    // Kernel launch
    _p.info() << "cuda kernel launching" << std::endl;
    checkCudaErrors(
        cuLaunchKernel( ~\label{code:executePtxWithParams:launch}~
            function, gridSizeX, gridSizeY,
            gridSizeZ, blockSizeX, blockSizeY,
            blockSizeZ, 0, NULL, 
            (void**)kernelParams.data(), NULL));

    // Cleanup
    checkCudaErrors(cuModuleUnload(cudaModule)); ~\label{code:executePtxWithParams:clean}~
    _p.info() << "cuda kernel executed successfully" << std::endl;

    return 0;
}
\end{lstlisting}