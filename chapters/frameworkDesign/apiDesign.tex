\section{API Design}
In this section we discuss the design of the API, and the reasoning that behind to these decisions.

\subsection{Goals}
As it is stated in the motivation in Section \ref{cha:motivation}, we want to prioritize high abstraction over absolute performance. To this end we want to avoid exposing kernel logic and put a layer of abstraction upon the general GPU model by handling kernel function setup and setup of blocks and threads for the developer.

In Section \ref{cha:languageSelection} it is stated that we want the API to be replaceable by more specific implementations written in \textit{CUDA}, as such it is preferred that the API allows an easier transition, by allowing interfacing with \textit{CUDA}.

There are multiple ways of achieving a higher level of abstraction and multiple ways of interfacing with \textit{CUDA}. In the following sections we discuss some of the possibilities and what decided to go forward with.

\subsection{Data types}
\todo{vi laver nogle data typer der indkapsler brug og data: arrays (ingen related), vectors(alle related), multidims(cppamp views)} 

\todo{skal de tilgås som stl containers(.begin() og .end())? ranges(bare .)? kompatibilitet med cuda(raw ptr)?}

We want to provide a developer with types that makes it convenient to work with the GPU. It is difficult to define what makes types accessible and easy to work with since there are multiple factors in play, such as the context of the programming language, the context of application, and what an individual developer sees as convenient. In this section and subsections, we explore some of these factors in the form of memory model, the actual data types, how types are accessed, and compatibility degree for CUDA, in order to determine how the types of \textit{YAGAL} should be designed.

\subsubsection{Memory Model} \label{memoryModelDesign}
% where is the data
Two general directions of handling the underlying memory of \textit{YAGAL}s data types. One is that the data types represent a unified memory layout, where the actual location of the data is handled by \textit{YAGAL}, involving any data transfers required to perform computations. The other is to provide methods for a developer to control the location of the data, giving her more work, but more explicit control. 

% Handle memory implicitly - Bolt and PACXX
In the case of \textit{YAGAL} being able to handle memory transfers between host and device implicitly, data should be available and behave as regular \textit{STL} data types when used on host, and be available for use on device. \textit{Bolt} and \textit{PACXX} handle memory this way, as seen in Chapter \ref{cha:relatedWorks}. An advantage of this approach is that a developer does not need to be concerned about memory when working with \textit{YAGAL}. A downside is that the developer loses control of when transfers are happening, which can come at a high performance cost. It may also create difficulties for a developer when she want to replace \textit{YAGAL} with \textit{CUDA}.

% Provide methods for memory handling - Thrust, AMP, and SkelCL
Memory in \textit{C++} is handled explicitly, and an experienced \textit{C++} developer is used to know and be in control of how and where data is stored. As \textit{YAGAL} is a \textit{C++} library, it makes sense to let developers handle memory themselves. Based on this we decided that \textit{YAGAL} will facilitate ways to manually manage where data is allocated and be able to control transfers between host and device. In Chapter \ref{cha:relatedWorks} we saw that \textit{Thrust}, \textit{C++ AMP}, and \textit{SkelCL} handles memory this way.

% The decision
We decided that \textit{YAGAL} will provide developers with the means to manage allocation and transfers of data between host and device themselves, instead of handling it implicitly.

\subsubsection{Supported Types}
% Introduction to considered types
There are many types that \textit{YAGAL} potentially can provide. To stay within the time frame of this project, we decide to keep the current amount of provided types to a minimum, and extend \textit{YAGAL} with additional types depending on the deadline. The types we have considered are \texttt{arrays}, \texttt{vectors}, and \texttt{maps}.

% We would like vectors, but arrays if it fails
\texttt{vectors} are contiguous memory that can be used as arrays with the possibility of being dynamically resized, in contrast to \texttt{arrays} which are contiguous memory with a static size. The frameworks in Chapter \ref{cha:relatedWorks} all provide \texttt{vectors} with the exception of \textit{C++ AMP} that uses \texttt{arrays} and \texttt{array\_views}. Based on this, we want \textit{YAGAL} to support \texttt{vectors}, but it is unclear to what degree of complications the implementation of \texttt{vectors} would introduce, due to their dynamic functionality. \texttt{arrays} on the other hand, appear to be more simple to implement due to them being statically sized. We will therefore focus on implementing \texttt{vectors} and have \texttt{arrays} as fallback solution, as the possibility of being able to dynamically resize a data collection is a quality that makes it easier for developers to use.

% Multi dim support
In regard to multidimensional support for \texttt{vectors} and/or \texttt{arrays}, it is convenient to have as it makes some tasks easier to implement, but it is of low priority as it is a more specialized feature, that is not needed by all developers, in contrast to single dimension arrays or vectors. We would useful for \textit{YAGAL} to support it, but it will not be prioritized.

% No support for maps
The \texttt{map} data container is an interesting option, since the GPU could be utilized to perform lookups. This however does not contribute to the overall goal of \textit{YAGAL}, as we have not observed this feature in any of the tools covered in Section \ref{related works} and the introduction of this feature does not contribute to \textit{YAGAL} being replaceable. \texttt{map} will therefore not be supported.

\subsubsection{CUDA compatibility}
\todo{snak om at det måske er godt at lade udvikleren gribe fat i lav niveau ting som device pointers, for at gøre transitionen nemmere senere.}

\subsection{Functions}
\todo{vi laver nogle højniveau funktioner, functors vs lambdas, chaining, vi kan snakke om mulighed for "collapsing", .execute()?}
\todo{gennemgå hvilke funktioner vi mener der er nødvendige}
\todo{map, reduce, filter, zip, access}
\todo{hvordan tager vi imod lambdaer}
