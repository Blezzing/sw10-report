\section{C++ AMP}
\textit{AMP} stands for Accelerated Massive Parallelism, and is a runtime library that allows a developer to write code to be executed on data-parallel hardware and is built upon \textit{DirectX 11}. \textit{C++ AMP} is developed by \textit{Microsoft} as a library and as an open standard for implementing parallelism in \textit{C++}. Their choice of \textit{DirectX 11} is interesting since \textit{OpenCl} and \textit{CUDA} existed at the time. The information discussed in this section was gained from \textit{Microsoft}'s \textit{C++ AMP} page \cite{microsoftCppAMP}.

% The commented text describes an implementation outputting OpenCL
% which have led to the announcement from the \textit{HSA Foundation} about an AMP compiler built with \textit{Clang}\cite{clang} and \textit{LLVM}\cite{llvm} that outputs to \textit{OpenCL} instead of DirectX11.

\subsection{Goals}
The aim of the \textit{C++ AMP} specification is to provide a way of writing code for data parallel hardware directly within the \textit{C++} language. \textit{Microsoft} implemented the specification based upon \textit{DirectX 11}, and the \textit{HSA Foundation} later did it for \textit{OpenCL}.

\subsection{Programming model}
A feature of \textit{C++ AMP} is that kernel functions is here expressed in \textit{C++} as restricted lambdas, meaning that a subset of \textit{C++} is available. 

Construction of matrices is done by first creating an array, and then wrap it with the \texttt{array\_view} that is provided by \textit{C++ AMP}. To show an example, an array is constructed below:
\begin{lstlisting}
int matrix[] = {1, 2, 3, 4}; 
\end{lstlisting}
To construct a matrix with two dimensions, the \texttt{matrix} array is wrapped with \texttt{array\_view}:
\begin{lstlisting}
array_view<int, 2> mat(2, 2, matrix); 
\end{lstlisting}
Here the \texttt{<int, 2>} specifies that the \texttt{mat} matrix consist of integers and two dimensions. \texttt{(2, 2, matrix)} indicates that the \texttt{mat} matrix will have two rows and two columns, and will be populated with the data from the \texttt{matrix} array;

Listing \ref{code:cppampSaxpy} shows \textit{SAXPY} implemented in \textit{C++ AMP}. The \texttt{array\_view}s are constructed at line \ref{code:cppampSaxpy:viewsStart} to \ref{code:cppampSaxpy:viewsEnd}. It is still needed to specify the views, even though this example only utilize one dimension. The \texttt{z\_v} \texttt{array\_view} is at line \ref{code:cppampSaxpy:discard} marked with the \texttt{discard\_data()} function. This is done to indicate that \texttt{z\_v} is used purely as an output container, and to avoid wasting resources transferring it to device since the contents will be overwritten.
At line \ref{code:cppampSaxpy:forEach} the function \texttt{parallel\_for\_each()} method is called and given two arguments. \texttt{z\_v.extend} indicates the compute domain. The lambda at line \ref{code:cppampSaxpy:lambda} are marked with \texttt{restrict(amp)} which states that the lambda should be executed on device and that only a subset \textit{C++} functionality is available for execution.
\begin{lstlisting}[caption={\textit{C++ AMP} \textit{SAXPY} example.}, label={code:cppampSaxpy}]
const size_t N = 1024;
int a = 10;

std::array<int, N> x;
std::array<int, N> y;
std::array<int, N> z;

std::generate(x.begin(), x.end(), rand);
std::generate(y.begin(), y.end(), rand);

array_view<const int, 1> x_v(size, x);~\label{code:cppampSaxpy:viewsStart}~
array_view<const int, 1> y_v(size, y);
array_view<int, 1> z_v(size, z);~\label{code:cppampSaxpy:viewsEnd}~
z_v.discard_data();~\label{code:cppampSaxpy:discard}~

parallel_for_each( ~\label{code:cppampSaxpy:forEach}~
    z_v.extent,

    [=](index<1> idx) restrict(amp){ ~\label{code:cppampSaxpy:lambda}~
        z_v[idx] = a * x_v[idx] + y_v[idx];
    }
)
\end{lstlisting}

\subsection{Implementation}

% Intro / Overview
\textit{C++ AMP} is a library that enables simple manipulation of large dimensional arrays by introducing a new language feature called \texttt{restricted} for \textit{C++}. 

% Memory / Types
Data are managed within regular \texttt{std::array}s. \texttt{array\_view}s are then constructed to represent these arrays as matrices on the GPU and used to manipulate them.

% Kernels / Fame?
Lambdas are used to construct the logic of a kernel for execution on the GPU. The \texttt{restricted} keyword are used in conjunction upon the lambda to indicate for the compiler that not all functionality are available within the scope of this lambda. 

To execute the kernel, the lambda are given to the \texttt{parallel\_for\_each} function that first copies the data described by the \texttt{array\_views} to the GPU and then executes the lambda.

% Comp / Figref
The compilation chain of \textit{C++} AMP can be seen in Figure \ref{fig:cppampCompilation}. Here the user code have included the \textit{C++ AMP} headers and compiled by the \textit{Microsoft Visual C++} compiler (\textit{MSVC}). The result is a executable file that utilizes \textit{Direct3D} to execute on the GPU. Even though \textit{C++ AMP} is an includeable library, it makes use of compiler specific functionality to add an additional keyword to the language and is therefore tied to the \textit{MSVC} compiler for \textit{windows}. The \textit{C++ AMP} specification is however open such that other compiler vendors could support it in the future.

\begin{figure}[H]
\center
\includegraphics[width=0.8\textwidth]{chapters/relatedWorks/figures/cppamp_compilation.png}
\caption{C++ AMP Compilation Process.}
\label{fig:cppampCompilation}
\end{figure}

\subsection{Key Points}
A unique feature of \textit{C++ AMP} is that it outputs to \textit{DirectX11}. We assume that this decision might have been made due to DirectX11 being developed and maintained by \textit{Microsoft} as well.

\textit{C++ AMP} is meant to extend \textit{C++} with parallelism. While this allows a developer to write code for GPUs in \textit{C++}, there are still some quirks in the form of \textit{array\_view} and restricted lambdas.
