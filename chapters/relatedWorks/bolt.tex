\section{Bolt}
\textit{Bolt} is a library providing abstractions for heterogeneous computing. This section is based on \textit{Bolt}'s documentation\cite{boltDoc} and \textit{Github} page\cite{boltGithub}.

\subsection{Goals}
The aim of \textit{Bolt} is to provide high performance library implementations for common algorithms, following the structure of \textit{STL}. It is intended to make heterogeneous development easier, and is designed to provide an application that can execute on either a CPU or any OpenCL capable unit.

\subsection{Programming Model}
\textit{Bolt} is modeled on \textit{STL} and as such, follows the model of calling functions with iterators as arguments to instruct where input and output is located.

\textit{Bolt} provide functions for modifying \textit{STL} containers, and the library determines whether the computation should happen on host or device, involving any required copying.

The example shown in Listing \ref{code:boltSaxpy}, shows how well the library interfaces with an \textit{STL} vector.

From line \ref{code:boltSaxpy:cppamp} until the next comment it is shown how the function is defined and implemented with the \textit{C++ AMP} backend. It is done with a \textit{C++11} lambda and \textit{C++ AMP}'s restrict classifier.

From line \ref{code:boltSaxpy:opencl} it is shown how, instead of a lambda, a functor is needed when using the \textit{OpenCL} backend. The functor is then defined inside a BOLT\_FUNCTOR macro to statically generate relevant \textit{OpenCL} code.

\begin{lstlisting}[caption={Bolt \textit{SAXPY} example}, label={code:boltSaxpy}]
const size_t N = 1024;
int a = 10;

std::vector<int> x(N);
std::vector<int> y(N);
std::vector<int> z(N);

std::generate(x.begin(), x.end(), rand);
std::generate(y.begin(), y.end(), rand);

//bolt with c++ amp backend ~\label{code:boltSaxpy:cppamp}~
auto saxpyLambda = [=] (float xx, float yy) restrict(cpu,amp) {
  return a * xx + yy;
};
bolt::transform(x.begin(), x.end(), y.begin(), z.begin, saxpyLambda);

//bolt with opencl backend ~\label{code:boltSaxpy:opencl}~
BOLT_FUNCTOR(SaxpyFunctor,
  struct SaxpyFunctor{
    int _a;
    SaxpyFunctor(int a): _a(a) {};
    float operator() (const int& xx, const int& yy){
      return _a * xx + yy;
    };
  };
);
boltcl::transform(x.begin(), x.end(), y.begin(), z.begin, SaxpyFunctor(a));
\end{lstlisting}

\subsection{Implementation}
\textit{Bolt} is a library on top of either \textit{C++ AMP} or \textit{OpenCL}\cite{boltPresentation}. The API is the same, while the supported features changes depending on which implementation is selected.
As shown in the example, only the \textit{C++ AMP} backend supports \textit{C++11} lambdas, which reduces the amount of work needed by the developer, compared to using a functor.

\subsection{Key Points}
Being able to use the same algorithms on both \textit{STL} containers, and \textit{Bolts} containers can provide an easier transition.

Being a library on top of other frameworks results in some code artifacts. With \textit{C++ AMP} as target, the use of the restrict classifier on lambdas seem unergonomic. \texttt{BOLT\_FUNCTOR} as a macro is used to overcome the language gap between \textit{C++} and \textit{OpenCL C} when \textit{OpenCL} is set as targeted backend. Both examples show that workarounds to support the target backend sometimes will show up in the API. 

The usage of the containers is based on iterators, as \textit{STL} is, and can be very verbose with multiple operations on the same container.
