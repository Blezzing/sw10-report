\section{Bolt} \label{cha:boltRelWorks}
\textit{Bolt} is a library providing abstractions for heterogeneous computing. This section is based on \textit{Bolt}'s documentation\cite{boltDoc} and \textit{Github} page\cite{boltGithub}.

\subsection{Goals}
The aim of \textit{Bolt} is to provide high performance library implementations for common algorithms, following the structure of \textit{STL}. It is intended to make heterogeneous development easier, and is designed to provide an application that can execute on either a CPU or any OpenCL capable unit.

\subsection{Programming Model}
\textit{Bolt} is modeled on \textit{STL} and as such, follows the model of calling functions with iterators as arguments to instruct where input and output is located.

\textit{Bolt} provide functions for modifying \textit{STL} containers, and the library determines whether the computation should happen on host or device, involving any required copying.

The example shown in Listing \ref{code:boltSaxpy}, shows how well the library interfaces with an \textit{STL} vector.

From line \ref{code:boltSaxpy:cppamp} until the next comment it is shown how the function is defined and implemented with the \textit{C++ AMP} backend. It is done with a \textit{C++11} lambda and \textit{C++ AMP}'s restrict classifier.

From line \ref{code:boltSaxpy:opencl} it is shown how, instead of a lambda, a functor is needed when using the \textit{OpenCL} backend. The functor is then defined inside a BOLT\_FUNCTOR macro to statically generate relevant \textit{OpenCL} code.

\begin{lstlisting}[caption={Bolt \textit{SAXPY} example}, label={code:boltSaxpy}]
const size_t N = 1024;
float a = 10;

std::vector<float> x(N);
std::vector<float> y(N);
std::vector<float> z(N);

std::generate(x.begin(), x.end(), rand);
std::generate(y.begin(), y.end(), rand);

//bolt with c++ amp backend ~\label{code:boltSaxpy:cppamp}~
auto saxpyLambda = [=] (float xx, float yy) restrict(cpu,amp) {
  return a * xx + yy;
};
bolt::transform(x.begin(), x.end(), y.begin(), z.begin, saxpyLambda);

//bolt with opencl backend ~\label{code:boltSaxpy:opencl}~
BOLT_FUNCTOR(SaxpyFunctor,
  struct SaxpyFunctor{
    float _a;
    SaxpyFunctor(float a): _a(a) {};
    float operator() (const float& xx, const float& yy){
      return _a * xx + yy;
    };
  };
);
boltcl::transform(x.begin(), x.end(), y.begin(), z.begin, SaxpyFunctor(a));
\end{lstlisting}

\subsection{Implementation}
% Intro / Overview
\textit{Bolt} is an abstraction library on top of either \textit{C++ AMP} or \textit{OpenCL}.  

% Memory / Types
The developers of \textit{Bolt} has made an effort in making it compatible with \textit{STL} types and algorithms. To this end, data can be automatically managed with \textit{STL} vectors, where data needed for a kernel will be copied to the device before execution, and back to the host after. \textit{Bolt} also provides device specific types for explicit memory management such as \texttt{device\_vector} which allocates data directly in device memory.

% Kernels / func
Lambda expressions are used to express kernel functions when \textit{C++ AMP} is the targeted backend. The procedure is similar to how it is done in \textit{C++ AMP} which is described in Section \ref{cha:cppAmpRelWork}, but \textit{Bolt} will handle the interaction with data and the \texttt{array\_view}s of \textit{C++ AMP} is therefore not needed. The \texttt{restricted} keyword is still used to tell the compiler that only a subset of functionality is available within the lambda scope.

Functors are used to express kernel functions when using \textit{OpenCl} as the targeted backend. A functor allows the construction of a struct that can be called as a regular function. 

% Comp / Figref
One of the two supported backends is \textit{C++ AMP}, and the compilation chain can be seen on Figure \ref{fig:boltAmpCompilation}. Here the user code has included the Bolt headers which in turn make use of the \textit{C++ AMP} headers. \textit{MVSC} is used for compilation since it is a compiler that supports \textit{C++ AMP}. This results in an executable file that utilizes \textit{Direct 3D} for execution on the GPU.

\begin{figure}[H]
\center
\includegraphics[width=0.8\textwidth]{chapters/relatedWorks/figures/bolt_cl_compilation.png}
\caption{Bolt Targeting OpenCL Compilation Process.}
\label{fig:boltClCompilation}
\end{figure}

The other of the two supported backends is \textit{OpenCL}, and the compilation chain can be seen on Figure \ref{fig:boltClCompilation}. It is the same procedure as with \textit{C++ AMP}, except that \textit{Bolt} makes use of OpenCL headers, it is not dependent upon the \textit{MVSC} compiler, and it  utilizes the OpenCL runtime.

\begin{figure}
\center
\includegraphics[width=0.8\textwidth]{chapters/relatedWorks/figures/bolt_amp_compilation.png}
\caption{Bolt Targeting C++ AMP Compilation Process.}
\label{fig:boltAmpCompilation}
\end{figure}

\subsection{Key Points}
\textit{STL} containers are more seamlessly integrated than they are in \textit{C++ AMP} since no \texttt{array\_view}s are needed. \textit{Bolt} can handle allocations for device and provides \texttt{device\_array} for cases where data should be manually managed.

As a library on top of other frameworks it shows some code artifacts of the underlying framework. With \textit{C++ AMP} as target, the use of the restrict classifier on lambdas is necessary. \texttt{BOLT\_FUNCTOR} as a macro is used to overcome the language gap between \textit{C++} and \textit{OpenCL C} when \textit{OpenCL} is set as targeted backend. Both examples show that workarounds to support the target backend sometimes will show up in the API. 