\section{Related Works}
/* meta here */

\subsection{Bolt}
/* meta here */
\subsubsection{Goals}
Bolt is designed to provide high performance library implementations for common algorithms, following the structure of STL. It is intended to make heterogeneous development easier.

It is designed to provide an application that can execute on either a CPU or any OpenCL capable unit.

\subsubsection{Implementation}
Bolt is a...

Bolt targets...

\subsubsection{Programming Model}
Bolt is modeled on STL and as such, follows the model of calling functions with iterators as arguments to instruct where input, and output is located.

An example is shown in Listing \ref{code:boltExample1}, where we sort a device vector. This is identical to the method shown in Listing \ref{code:boltExample2}, where a a std::vector is sorted with std::sort.
\begin{lstlisting}[caption={Bolt sort example}, label={code:boltExample1}]
//vector construction
bolt::cl::device_vector<int> input(1024);

//vector fill ommited

//inplace sort
bolt::cl::sort(input.begin(), input.end());
\end{lstlisting}

\begin{lstlisting}[caption={STL sort example}, label={code:boltExample2}]
//vector construction
std::vector<int> input(1024);

//vector fill ommited

//inplace sort
std::sort(input.begin(), input.end());
\end{lstlisting}



\subsubsection{Cognitive Dimensions of Notations}

