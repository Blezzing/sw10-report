\section{Problem Statement}
Based on our motivation in Section \ref{cha:motivation}, we describe our goal as a problem statement. Then we specify which tasks we need to perform to achieve it, and delimit our goal.

%new one
\textit{Can a GPGPU framework to abstract the underlying programming model be created, as a library that does not limit the developers choice of compiler, and
how does it compare to other frameworks that do?}

%"How can we",and if we can't, "Can we"
%old one:
%\textit{How can we create and evaluate a framework, that allows developers without GPGPU development experience to gain initial benefit from GPU acceleration, before learning about a low level GPGPU programming model?}

\subsection{Tasks} \label{cha:tasks}
To reach the goal described in our problem statement we have defined the following tasks:

%old
%\begin{description}
%\item[Create overview of related works] \hfill \\
%As others have made attempts at making GPGPU development simpler, we can learn from their design choices and results.
%\item[Research framework design principles] \hfill \\
%Identify vocabulary and good practices for developing frameworks.
%\item[Design framework] \hfill \\
%Design the architecture and API of the framework.
%\item[Setup for implementation of framework] \hfill \\
%Configuring environment and performing necessary preliminary tasks for implementation.
%\item[Implement framework] \hfill \\
%Following the design.
%\item[Research how to compare frameworks] \hfill \\
%To be able to review our framework in perspective of related works.
%\item[Develop demo applications] \hfill \\
%To showcase features and differences compared to related works, based on the research of framework comparison methods.
%\item[Evaluation of our framework] \hfill \\
%To review how our framework compares to the related works, and how well it reached our goal.
%\end{description}

\begin{description}
\item[Create overview of related works] \hfill \\
As others have made attempts at making GPGPU development simpler, we can learn from their design choices and results.
\item[Research framework design principles] \hfill \\
Identify vocabulary and good practices for developing frameworks.
\item[Design framework] \hfill \\
Design the architecture and API of the framework.
\item[Setup for implementation of framework] \hfill \\
Configuring environment and performing necessary preliminary tasks for implementation.
\item[Implement framework] \hfill \\
Attempting to achieve the design and experiencing what problems, and possible solutions, that arise from limitations.
\item[Develop demo applications] \hfill \\
To showcase features what is possible in the framework, and to compare with implementations in related works.
\item[Evaluation of our framework] \hfill \\
To review how our framework compares to the related works, and what the costs of the limitations are.
\end{description}

\subsection{Delimitation}
The problem can be approached in various ways, but the following are not the priority of the project:

%old
%\begin{description}
%\item[Outperforming CUDA/OpenCL] \hfill \\
%While we want the framework to perform well, we do not attempt to reach better performance than CUDA and OpenCL, as this would be unfeasible due to costs of abstraction.
%\item[Being equivalent in features to any other library] \hfill \\
%While we want the library to be usable in as many use cases as possible, we do not attempt to be able to do everything for everyone, so some features known from other libraries, might not be implemented here.
%\end{description}

\begin{description}
\item[Outperforming CUDA/OpenCL] \hfill \\
While we want the framework to perform well, we do not attempt to reach better performance than CUDA and OpenCL, as this would be unfeasible due to costs of abstraction.
\item[Being "Complete"] \hfill \\
While we want to provide the functionality to allow a developer to develop GPGPU applications, we prioritize investigating approaches to work around not using a special purpose compiler, and as such the development efforts will not be focused on the completeness of the framework, but rather experimentation.
\end{description}