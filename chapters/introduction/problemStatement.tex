\section{Problem Statement}
Based on our motivation in Chapter \ref{cha:motivation}, we describe our goal as a problem statement. Then we delimit it, and specify which tasks we need to perform to achieve it.

\textit{How can we create a framework, that allows developers without GPGPU development experience to gain initial benefit from GPU acceleration, before learning about a low level GPGPU programming model?}

\subsection{Delimitation}
There are certain details in the problem, that can be expanded upon to much more elaborate solutions than what we will do in this project. They are:
\begin{description}
\item[Outperforming CUDA/OpenCL] \hfill \\
While we do want to be performant, we do not attempt to be better performance wise than CUDA and OpenCL, as this would be unfeasable due to time constraints and the costs of abstraction.
\item[Being equivalent in features to any other library] \hfill \\
While we want the library to be usable in as many use cases as possible, we do not attempt to be able to do everything for everyone, so some features known from other libraries, might not be implemented here.
\end{description}

\subsection{Tasks} \label{cha:tasks}
To accommodate the project statement we have defined following tasks.
\begin{description}
\item[Analyze related works] \hfill \\
As others have made attempts at making GPGPU development simpler, we can learn from their design choices and results.
\item[Research framework design principles] \hfill \\
Identify vocabulary and good practices for developing frameworks.
\item[Design framework] \hfill \\
Define how the framework should be.
\item[Implement framework] \hfill \\
Following the design as close as possible.
\item[Develop demo applications] \hfill \\
To showcase features and differences compared to related works.
\item[Research how to compare frameworks]
To be able to review our framework in perspective of related works.
\item[Comparison of frameworks] \hfill \\
To review how our framework compares to related work.
\end{description}
\todo[inline]{denne liste er meget wip.}
\todo[inline]{hvordan får vi det med i problemformuleringen at vi skal sammenligne med andre?}
