\section{Anonymous Functions}
We consider anonymous functions as an important part of a framework that allows construction of other functions, which in the case of \textit{YAGAL} refers to the construction of GPU kernels. Implementing support for anonymous functions in \textit{YAGAL} to enable developers to extend functionality is not a straight forward task, and different approaches are discussed in this section. The code examples shown in this section are not supported by \textit{YAGAL}, but show how anonymous functions could look as a result each of our proposed approaches. The examples present a predicate, checking whether a squared value of a vector element is above 1000, which should be given to other non-existant \textit{YAGAL} functions such as \texttt{filter} to provide their logic.

\subsection{Known Approaches}
%c++ lambdas
An initial thought is that \textit{C++11} provide expressive lambdas, which seem ideal for a framework such as \textit{YAGAL}. The problem with expanding upon this construct is that it requires a compiler to be built in order to determine how to handle the lambda body, as the body is not available for a library to inspect\cite{cppLambdaRef}, and an important part of the lambda is the body, since that is the part needed for code generation. In the related works we see that \textit{C++ AMP} in section \ref{cha:cppAmpRelWork} and \textit{Thrust} in section \ref{cha:thrustRelWork} both take the approach of using extended \textit{C++} lambdas. In both cases they have extended the language with keywords in order to annotate the lambdas, and thereby tell their compiler know how the lambda should be treated differently. Even though \textit{PACXX} lambdas can be defined with clean \textit{C++} syntax, it still requires a specific compiler in order to generate kernels that are based upon them. In \textit{YAGAL} we chose not to enforce the compiler choice on the developer, and thereby rendering this approach as not possible.

%opencl kernel strings
Another method, which is used by \textit{SkelCL} as described in section \ref{cha:skelclRelatedWorks}, and \textit{Bolt} on \textit{OpenCL} as described in section \ref{cha:boltRelWorks}, is to generate a kernel as a \textit{OpenCL} string by concatenation of strings. This approach involves user defined strings that describes the anonymous function. A problem with the approach is that it requires the framework to be based upon OpenCL, whereas \textit{YAGAL} is based on \textit{CUDA}. With \textit{OpenCL} being syntactically very similar to \textit{C++}, it is a usable solution for allowing a developer to specify lambdas. In contrast, providing similar functionality, where the user provides the lambda as a \textit{PTX} string, is far less intuitive due to the difference between \textit{C++} and \textit{PTX}.

%reetablishing that this is a beepin problem
As we are not based on \textit{OpenCL}, and we do not provide a special purpose compiler, none of the related works provide a model we can re-use in \textit{YAGAL}.

\subsection{Alternative Approaches}
To work around this problem we have tried to rethink the way kernel logic can be constructed. We came up with two ideas for constructing it. One approach is to extend the library with meta types that represent logic that would otherwise have been provided through lambdas. Another approach is to create a \textit{CUDA C} string, and use external tools to compile it to \textit{PTX} at run-time. These approaches are discussed in the following subsections.

\subsubsection{Creating Meta Types}
A function body could be represented by objects representing the expected syntax. An example of what this could look like is shown in listing \ref{code:metaBuildExampleC} as being represented with meta types in listing \ref{code:metaBuildExampleBuild}. The different components of the kernel can be constructed like the current action implementation described in chapter \ref{cha:frameworkImplementation}, allowing \textit{LLVM Intermediate Representation} to be generated. 

\begin{lstlisting}[caption={Pseudo C code showing a kernel function.}, label={code:metaBuildExampleC}]
bool pred(float element){
    return element * element > 1000f;
}
\end{lstlisting}

\begin{lstlisting}[caption={Code showing how construction of kernel function in a meta type solution could be done.}, label={code:metaBuildExampleBuild}]
auto builder = KernelBuilder.New<bool, float>();
builder.addParameter(Parameter<float>("element"));
builder.addStatement(ReturnStatement<bool>(
    GreaterThanComparison<float>(
        builder.multiply<float>(
            builder.getValue<float>("element"), 
            builder.getValue<float>("element")
        ), 
        builder.createConstant<float>(1000)
    )
));

auto pred = builder.getResult();
\end{lstlisting}

This solution is extremely verbose, and difficult to read, compared to the logic that is being constructed. A developer would have an easier time by simply using another framework. Implementation wise it is a big task to implement all the different kinds of operations that would be possible. If the wrapping of logic can be achieved in a more intuitive and readable way, it can be a feasible solution.

\subsubsection{Wrapping by String Interpretation}
The meta types presented in the previous section in listing \ref{code:metaBuildExampleBuild} could be the result of a parser. The parser could parse a well formatted \textit{C++} string, and create the objects based upon the content of the string. With this approach it would be simpler to make use of existing compiler techniques, such as utilizing a syntax tree, compared to using a kernel builder as shown on \ref{code:metaBuildExampleBuild}. This tree could contain all information needed for performing  code generation with \textit{LLVM}, but would practically mean that \textit{YAGAL}, on top of integrating a considerable part of \textit{LLVM}, need to include other components of a compiler, that being a lexer, a parser, and syntactical objects for building syntax trees.

A function that perform the same calculation as the above meta type example, could with such a solution look like listing \ref{code:metaBuildString}. It is similar to the example shown in \ref{code:metaBuildExampleC}, which is what makes it attractive. 

\begin{lstlisting}[caption={Code showing possible construction of kernel with string interpretation.}, label={code:metaBuildString}]
auto pred = yagal::predicateFromString<bool, float>(
    "bool pred(float element){
        return element * element > 1000f;
    }"
);
\end{lstlisting}

Even though the solution in listing \ref{code:metaBuildString} looks more write- and readable than the example shown in listing \ref{code:metaBuildExampleBuild} it is not a solution without problems. Requiring compilation of the string to be done at run-time will make it more difficult to get errors or warnings presented to the developer, when compared to static compilation. There is also the minor annoyance of having strings in user code that provide functionality, as it generally makes it difficult for tools, such as an IDE, to assist the developer in writing and checking code, e.g. by providing syntax highlighting or error detection.

\subsubsection{Wrapping by C-Style Macros}
Bolt, targeting \textit{OpenCL}, use \textit{C} macros to wrap function objects. This allow it to have the function, after the macros interaction, syntactically checked by the compiler, as the result would still be code. It also allows it to generate an \textit{OpenCL} code string based upon this function. As \textit{C++} and \textit{OpenCL} is syntactically similar in how a function is constructed, the major part of code generation is done by copying the content of the macro to a string constant. If such a solution should work for \textit{YAGAL}, it could be done by targeting one of the other approaches, such as the previously mentioned string interpreter, by constructing a string for that.

\begin{lstlisting}[caption={Code showing possible construction of kernel with macro, named YAGALIFY, and \textit{C++} lambda.}, label={code:metaBuildMacro}]
auto pred = YAGALIFY(
    bool pred(float element){
        return element * element > 1000f;
    }
);
\end{lstlisting}

This solution can be considered cleaner than using strings only, but it does add a layer of complexity to the already high stack.

\subsubsection{Generating CUDA C}
A different approach, compared to the previously mentioned approaches, is to let a developer define a function in a string, and wrap that function in the needed components to make it a valid \textit{CUDA C} function. That function can then be sent to the external tool \textit{nvrtc}\cite{nvrtcDoc}, \textit{NVIDIA Run-time Compiler}, which generates the \textit{PTX} code needed for that function. Even though this approach requires an additional external dependency, it does have the benefit solving the task without inventing a new language and the means of compiling it, as the previous examples did.

Example usage can is shown in listing \ref{code:metaBuildNvrtc}. Notable is the lack of \textit{CUDA} keywords, such as \texttt{\_\_DEVICE\_\_} to declare that it is a device function. This can be inserted by the library before parsing it to \textit{nvrtc}, along with replacing other keywords we could introduce with intrinsics. This would relieve the user of having to learn some of the \textit{CUDA} specifics involved when writing kernels.

\begin{lstlisting}[caption={Code showing a possible construction of a kernel based on a string. The string is being sent to the library, where it get extended to correct \textit{CUDA C}, before being sent to \textit{nvrtc}.}, label={code:metaBuildNvrtc}]
auto myKernel = yagal::ptxFromCudaString(
    "bool pred(float element){
        return element * element > 1000f;
    }"
);
\end{lstlisting}

\subsection{What Is Implemented in YAGAL}
In \textit{YAGAL} we have no implementation of anonymous functions, or functions using them. We feel that there is a need for them, but the approaches are too involved for the time constraints of this project. The meta type solution is more straight forward to generate \textit{LLVM Intermediate Representation} for, but would require too much development time, and it would be inconvenient for a developer to express functions with. The ideas of parsing strings, or using macros, to make the expressions less verbose could alleviate the problems of the first method, but will still require too much development time. The final solution, using \textit{nvrtc} might be the one that is the most fitting for \textit{YAGAL}, as we already rely on executing \textit{PTX}. It does however come with the cost of a tighter coupling with \textit{CUDA}.

When a method of implementing anonymous functions is done, implementing functions to use them would be possible.